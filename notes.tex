\documentclass[11pt]{article}
\usepackage{graphicx}
\usepackage{fancyhdr}
\usepackage{multicol}
\usepackage{listings}
\usepackage[margin=1in]{geometry}
\usepackage[colorlinks=true, linkcolor=black, urlcolor=cyan]{hyperref}

\begin{document}

\begin{titlepage}
\begin{center}
%\includegraphics[height=3.5cm]{images/Standard_CUAUV.jpg}\\[0.2cm]
\textsl{\huge MIT Splash}\\[0.5cm]
{\huge Fall 2014}\\[0.2cm]
\rule{\linewidth}{0.5mm}\\[0.2cm]
{\Huge Machine Learning and Audio Analysis with Python}
\rule{\linewidth}{0.5mm}\\[0.4cm]
\huge Course Notes\\[0.2cm]
\large Daryl Sew
\end{center}
\end{titlepage}

% set up header and footer
\pagestyle{fancy}
\fancyhf{}
\setlength{\headheight}{30pt}
\renewcommand{\headrulewidth}{0.4pt}
\renewcommand{\footrulewidth}{0.4pt}
%\lhead{\includegraphics[height=8mm]{images/Standard_CUAUV.jpg}}
\rhead{Machine Learning and Audio Analysis with Python}
\rfoot{Fall 2014}
\cfoot{\thepage}

% Table of contents
\tableofcontents
\pagebreak

\section{Overview}

Machine learning is a field of computer science that concerns writing programs that can make and improve predictions or behaviors based on data inputs. The applications of machine learning are very diverse - they range from self driving cars to spam filters to autocorrect algorithms and much more. Using scikit-learn, an open source machine learning library for Python, we'll cover reinforcement learning (the kind used to create artificial intelligence for games like chess), supervised learning (the kind used in handwriting recognition), and unsupervised learning (the kind eBay uses to group its products). We'll then cover audio analysis through Fourier transforms with numpy, an open source general purpose computational library for Python, and we'll use our newfound audio analysis and machine learning skills to write very basic speech recognition software. Applications of machine learning to the fields of multitouch gesture recognition and computer vision will also be discussed, drawing from my work at Tesla and research on self driving cars and autonomous submarines.
%\begin{figure}[h!]
%    \centering
%%        \includegraphics[scale=0.5]{images/calibration.png}
%        \caption{\texttt{StereoCalibModule} recognizing a usable frame for calibration}
%    \end{figure}
%\\

\section{What is Machine Learning?}
All machine learning algorithms aim to take observations from a system and produce a model of that system. The key words here are 'system' and 'model'. Systems include anything from stock markets to populations of organisms to the environment surrounding a robot, essentially anything imaginable for which you can record observations. Models are a set of mathematical rules that describe a system. 

\section{Unsupervised Learning}
\subsection{K-means clustering}
\subsubsection{Theory}
\subsubsection{Applications}
1-2 examples w code

\section{Semi-supervised Learning}

\section{Supervised Learning}
\subsection{Regression}
Regression is useful for creating a continuous model of a system based on inputs and outputs.
\subsubsection{Theory}
The derivation for linear regression is actually fairly straightforward.

\subsection{Support Vector Machines}
\subsubsection{Theory}
\subsubsection{Applications}

\section{Reinforcement Learning}
box2d.js
\subsection{Theory}
\subsubsection{Applications}

\section{What is Audio?}

\end{document}
